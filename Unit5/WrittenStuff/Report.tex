\documentclass[12pt]{article}
\usepackage{graphicx}
\usepackage{amsmath}
\usepackage{float}
\usepackage{geometry}
\geometry{margin=1in}
\usepackage[colorlinks=true, linkcolor=blue, urlcolor=blue, citecolor=blue]{hyperref}
\title{Cat Food Analysis of Variance}
\author{Nathan Yan}
\date{\today}

\begin{document}

\maketitle

\begin{abstract}
Data was collected for many manufacturers to determine whether there was a difference between mean price of wet cat food between different companies. We find that these results are significant, though there are some issues with assumptions that require that one proceeds with caution with this conclusion. 
\end{abstract}

\section{Introduction}
This is my 4th six weeks project for Data Science, for Unit 5: Analysis of Variance. There are many different cat food companies, all with different manufacturing processes and quality of cat food. I was wondering one day if there was a difference between average cost across all of these companies. I hypothesized that because these cat food companies speciallized in different quality foods (for example, \emph{Fancy Feast} was known for their more expensive, higher quality cat food), the mean prices for wet cat food for these companies would be different. This is important because the mean price of cat foods are a key aspect that drives consumer decisions about what cat food to purchase.

\section{Data Description}
I limited scope to wet cat food because dry food can potentially be 5 to 10 times cheaper than wet food, which would likely result in extremely high variances. All data was collected from \url{https://www.chewy.com/} which sells cat food from many different companies. I collected data from 5 different companies, \emph{Fancy Feast}, \emph{Chewy}, \emph{Blue Buffalo}, \emph{Sheba}, and \emph{Applaws}. For each company I randomly selected a few product offerings from the website, collecting data on manufacturer, price, weight in ounces, and rating. I computed the price per ounce of each product from this data. In total there are 26 data points, with each company having about 5 data points.

\section{Methodology}
ANOVA is the optimal method for this data because it contains many categorical groups defined using quantitative data. Price itself is extremely strongly correlated with the weight in ounces, so to reduce the effect of this confounding variable, we use price per ounce. Although there are concerns that price per ounce is still influenced by the total price, an ANOVA over the linear regression shows that it is no longer significant ($p = 0.51$).

We have the following hypotheses:
\[H_0: \text{All cat food brands have the same average price per ounce}\]
\[H_a: \text{At least one brand has a different mean price per ounce}\]

\subsection{Assumptions}

The randomness condition holds because I have chosen the products at random. Independence is fairly certain to be true here. Comparing standard deviations, we find that 

\[\frac{\max(s_i)}{\min(s_i)} = 2.230864\]

This is just over 2, which is the bound we usually set for the homoscedasticity condition, but since it's so close I decided to accept it anyways. However, I proceed with caution.

Its important to note that because it is an observational study, causality cannot be concluded. However, I have done my best to reduce confounding variables as much as possible. 

\begin{figure}[H]
    \centering
    \includegraphics[width=0.8\textwidth]{assets/qq.png}
    \caption{Although there are some notable outliers on the right tail of the distribution, there are few enough that we can assume the non-normality condition holds.}
\end{figure}

\begin{figure}[H]
    \centering
    \includegraphics[width=0.8\textwidth]{assets/residuals.png}
    \caption{While there are a few outliers, this plot still looks pretty decent.}
\end{figure}


\section{Results}
After running an ANOVA test, we get the following table:
% latex table generated in R 4.5.2 by xtable 1.8-4 package
% Sun Feb  1 17:11:14 2026
\begin{table}[ht]
\centering
\begin{tabular}{lrrrrr}
  \hline
 & Df & Sum Sq & Mean Sq & F value & Pr($>$F) \\
  \hline
Manufacturer & 4 & 0.83 & 0.21 & 4.47 & 0.0090 \\
  Residuals    & 21 & 0.97 & 0.05 &  &  \\
   \hline
\end{tabular}
\end{table}

Because $p = 0.009 < 0.5$, we reject the null hypothesis and conclude that at least one of the cat food brands has a different mean price from the others.

Since our ANOVA was significant, we can conduct some post-hoc tests. Applying an Fisher-LSD test with a bonferroni-corrected p value, we found that Applaws is significantly different from Blue Buffalo and Chewy. Without the bonferroni correction, Applaws is significantly different from \emph{all} the other brands, though the group-wise error rate cannot be guaranteed to be less than $0.05$. 
% \vspace{1em}
% latex table generated in R 4.5.2 by xtable 1.8-4 package
% Sun Feb  1 17:23:04 2026
\begin{table}[ht]
\centering
\begin{tabular}{rrl}
  \hline
 & PricePerOz & groups \\
  \hline
Applaws & 0.91 & a \\
  Sheba & 0.57 & ab \\
  Fancy Feast & 0.49 & ab \\
  Blue Buffalo & 0.45 & b \\
  Chewy & 0.31 & b \\
   \hline
\end{tabular}
\end{table}


\section{Conclusion}
Applaws' mean price per ounce is much, much higher than all the other companies that I checked. This makes, sense, since they are the most luxury brand on the list, with all fancy matte-black packagings and whatnot. All the other brands were not significantly different from each other, even without the bonferroni correction, suggesting that they are reletively similiar in terms of target audience price-wise. 


\end{document}